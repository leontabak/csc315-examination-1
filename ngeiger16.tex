
\chapter{Nicci Geiger}

\begin{enumerate}
  \item Most programming languages require the use of brackets to
    enclose the index in a reference to an element of an array.
  \begin{enumerate}
    \item Identify a language the requires the use of parentheses
      to enclose the index in a reference to an element of an array.
    \item Why did the designers of the language choose parentheses
      rather than brackets?
    \end{enumerate}

  \begin{answer}

  \begin{enumerate}
    \item Ada uses parentheses to enclose the index refrence to the
      element of an array.
    \item This was done for uniformity between array refrences and
      function call expressions because they are both mappings.
    \end{enumerate}

    \end{answer}
    
  \item What is the relationship between a lexeme and a token?

  \begin{answer}

    A lexemes is the lowest-level syntatic units.  Lexemes are
    patitioned into groups which are represented by tokens.
    \end{answer}

  \item
  \begin{enumerate}
    \item What kind of symbols are found at the internal nodes of a
      parse tree?
    \item What kind of symbols are found at the leaves of a parse tree?
    \end{enumerate}

  \begin{answer}

  \begin{enumerate}
    \item Nonterminal symobols are found at internal nodes.
    \item Terminal symbols  are found at leaves.
    \end{enumerate}

    \end{answer}


  \item One of the most significant contributions from the developers
    of ALGOL 60 also limited the success of that language. What was
    that contribution?

  \begin{answer}

  The contribution was BNF (Backus-Naur Form) a natural notation for
  describing syntax.

    \end{answer}

  \item What problem were the creators of Common LISP trying to solve?

  \begin{answer}

    They were trying solve the issue of lack of portability in
    programs written in various dialects.

    \end{answer}

  \item What is an ambiguous context free grammar?

  \begin{answer}

   A grammar that generates a sentential for that has two or more parse trees.

    \end{answer}

  \item Contrast the complexity of algorithms that can parse strings
    that conform to the most general kinds of context free grammars
    and the complexity of the algorithms that can parse strings that
    conform to the grammars of programming languages?

  \begin{answer}

    The complexity of context free grammars is $O(n^3)$, while the
    grammars of programing languages have a complexity of $O(n)$ due to
    the fact that context free grammar parsers must frequently be
    backed up and reparse part of what is being analyzed.

    \end{answer}

  \item Java represents characters with Unicode. It is the first
    widely used programming language with this feature. What is the
    significance of this feature?

  \begin{answer}

   It uses a set of charactes that covers characters for most of the
   worlds alphabet.  natural languages. This caters to the need for
   global computer communication.

    \end{answer}

  \item How does the binary coded decimal type differ from the
    floating point type?

  \begin{answer}

    Float points type represent real numbers but are only approximate
    and are stored in binary and represented in fractions or
    exponents, while decimal type stores a fixed number of decimal
    points and precicely store them with a restricted range.

    \end{answer}

  \item Identify a user-defined ordinal type in the Java programming
    language.

  \begin{answer}

    Java has enumeration types so long as it is after Java 1.4.

    \end{answer}

  \item Mathematicians and programmers might have different ideas
    about the precedence of Boolean operators. Explain.

  \begin{answer}

   Mathematicians have the OR and AND opperators with equal precedence
   while programmers have AND with a higher precedence than OR.

    \end{answer}

  \item Programmers should use \verb+===+ rather than \verb+==+ to
    test the equality of the values of two expressions in JavaScript. Why?

  \begin{answer}

    The \verb+===+ over \verb+==+ prevents coersion of one type into
    another such as a string that is numbers to be coerced into a
    number rather than being a string containing the number.

    \end{answer}

  \item Describe a hazard of allowing short-circuited evaluation
    of expressions and side effects in expressions at the same time.

  \begin{answer}

    Short-circuit evaluations used in an expression with part of the
    expression being a side effectthat is not evaluated. The side
    effect will only occur in complete evaluation of the whole
    ezxpression. This causes the side effects in expressions to allow
    for subtle errors.

    \end{answer}

  \item Briefly describe the three steps in the mark-sweep algorithm
    for garbage collection.

  \begin{answer}

   The first are cells in the heap have indicators set to garbage. The
   second is when every pointer in the program is traced to the heap
   and reachables are marked to not be garbage. The third is that all
   cells not marked as being used are returned to being useable space.

    \end{answer}

  \item What led Yukihiro Matsumoto to create the Ruby programming language?

  \begin{answer}

    There was a dissatisfaction with Perl and Python and the designers
    as both supported object oriented programing but neither were
    purely object oriented programing.

    \end{answer}

  \item What did Microsoft aim to achieve with its development of the
    C\# language?

  \begin{answer}

    C\# was aimed to be a component-based software development in the
    .NET framework.

    \end{answer}

  \end{enumerate}


\section{More questions for discussion and review.}

\begin{enumerate}
  \item The design of which machine influenced the design
    of the control statements in FORTRAN?
    
 \begin{answer}
    IBM 704 system 
     \end{answer}

  \item How many different kinds of control statements
    must the designer of a programming language include
    in a language?


 \begin{answer}
    While it is possible to use only one (GOTO), the minimum number of control statements in a language which does not use GOTO is two. One for choosing between control flow paths, and one for logically controlled iterations. (348)
 \end{answer}
 
  \item What is the one question that applies in the
    design of all statements that allow selection or
    iteration?

 \begin{answer}
    Should the control structure have multiple entries? (349)
 \end{answer}
 
  \item What is an advantage of requiring that
    the \textbf{then} and \textbf{else} clauses of
    an \textbf{if} statement be compound statements?

  \item How does the \textbf{switch} statement in C\#
    differ from the \textbf{switch} statement in Java?

  \item Distinguish between 2 statements in Ruby
    that correspond to Java's \textbf{switch} statement.

 \begin{answer}
    Requiring compound statements helps increase the readability and writability for programmers when using nested selector statements, that otherwise can get very messy and complicated.
 \end{answer}
 
  \item How does the \textbf{switch} statement in C\#
    differ from the \textbf{switch} statement in Java?

 \begin{answer}
   In Java, the switch statement does not allow case expressions anywhere except the top level in the body of the switch.  C\#  allows the execution of more than one segment. In C\#, the control expressions as well as case statements can be strings. (357)
 \end{answer}
 
  \item Distinguish between 2 statements in Ruby
    that correspond to Java's \textbf{switch} statement.

 \begin{answer}
    Case expressions are the Ruby equivalent of Java's \textbf{switch} statement. One is semantically similar to nested if statements with case - when - then. The other is with boolean expressions being evaluated one at a time from top to bottom .The value of this case expression ends up as the first value with a true corresponding boolean. (357-358)
     \end{answer}

  \item Features of a programming language sometimes persist
    longer than a feature of computing hardware that inspired
    and supported that part of the language's design.
    Similarly, features of hardware sometimes persist longer
    than some parts of a language's design that were created
    to take advantage of that feature in hardware.

    Give examples.

 \begin{answer}
    The IBM 704 influenced the design of control statements which are still used today, and prompted the development of Fortran. (348, 42)
    Most of the popular languages of the past 50 years have been designed around the von Neumann architecture. These are called imperative languages. In a von Neumann computer, both data and programs are stored in the same memory. The CPU, which executes instructions, is separate from the memory. Therefore, instructions and data must be transmitted, or piped, from memory to the CPU. Results of operations in the CPU must be moved back to memory. Nearly all digital computers built since the 1940s have been based on the von Neumann architecture.
The Algol standard used several different syntaxes which, among other things, allowed Europeans to use a comma to denote a decimal point, while Americans could continue to use a period.
  The \textbf{register} keyword in C is a hint to the compiler that a variable will be used repeatedly, and so it should be stored in the CPU rather than in memory. However, modern compilers are far better at optimization than programmers, so this keyword is outdated and unnecessary.
     \end{answer}

  \item Who most famously warned of the dangers of using the
    \textbf{goto} statement? What did Donald Knuth have to
    say about the use of the \textbf{goto} statement?

 \begin{answer}
    Edsger Dijkstra noted “The goto statement as it stands is just too primitive; it is too much an invitation to make a mess of one’s program.” Donald Knuth argued there were occasions when the efficiency of the goto outweighed its harm to readability. (376)
     \end{answer}

  \item Describes Ada's \textbf{for} loop. Are there some
    kinds of iteration that might be easier in Ada than
    in Java? Easier in Java than in Ada?

  \item What does it mean to say that the guarded commands
    of Ada are non-deterministic?

  \item The header files in a C program contain function
    prototypes. What is a function prototype?

 \begin{answer}
    The for loop looks like this
\begin{lstlisting} for variable in [reverse] discrete\_range loop … end loop; /end{lstlisting}
Ada’s for loop can use any ordinal type variable for its counter. Arrays with ordinal type subscripts can be conveniently processed. (261)
   \end{answer}
   
  \item What does it mean to say that the guarded commands
    of Ada are non-deterministic?

 \begin{answer}
    Guarded commands in Ada are nondeterministically chosen for execution when more than one of the statements are evaluated to true. This means that if there are three guarded statements and two of the three evaluate to true, then each time the program will use one of the two statements. It will not always use the one that appears first nor the one that appears last, but rather it will choose between them non-deterministically or randomly.
     \end{answer}

  \item The header files in a C program contain function
    prototypes. What is a function prototype?

 \begin{answer}
   A function prototype is a function declaration that gives the function's name and type signature, but does not specify the function body. It is also referred to as a function interface at times. In other languages theses are uncommon because subprograms do not need declarations since they do no need to be defined before they are called.
 \end{answer}

  \item Every method in a Ruby program belongs to a class.
    A programmer can place a definition of a method inside
    the definition of a class or outside of the definition
    of any class that the programmer writes. To which class
    does the method belong in the second case?

  \item Distinguish between positional and keyword parameters.

  \item Ruby blocks are closures. What does that mean?

  \item What is a pure function?

  \item Some languages give programmers means to define
    both functions and procedures. Java doe not. Is that
    a serious limitation?

 \begin{answer}
     If a method is defined outside of the definition of any class that the programmer writes then the method belongs to the root object, \textbf{Object}.
     \end{answer}

  \item Distinguish between positional and keyword parameters.

 \begin{answer}
    Positional parameters are bound based on the order in which they are given to the function. Keyword parameters are when the name of the formal parameter to which an actual parameter is to be bound is specified with the actual parameter in a call. This allows them to be input in any order
     \end{answer}

  \item Ruby blocks are closures. What does that mean?

 \begin{answer}
     A closure is an anonymous function which can be passed as a parameter. Also, Ruby doesn't nest scope, so variables defined in a method are not accessible outside of the method (say, in the class which defines them).
     \end{answer}
   

  \item What is a pure function?

\begin{answer}
    A pure function is a function where the return value is only determined by its input values, without observable side effects.
\end{answer}

  \item Some languages give programmers means to define
    both functions and procedures. Java does not. Is that
    a serious limitation?

\begin{answer}
    It does not seem to be a serious limitation, it just means that performing procedures in Java is more verbose because it requires that the programmer instantiate the class with any needed methods.
    \end{answer}

  \item Declarations of formal parameters in an Ada procedure
    can include, in addition to the names and types of the
    parameters, reserved words that do not appear in declarations
    in Java programs. 
    What is the purpose of those reserved words?
 
 \begin{answer}
    Ada allows the programmer to specify in mode, out mode, and inout mode for each formal parameter.  This means that they can receive data from the actual parameter, they can send data to the actual parameter, or they can do both.
    \end{anwser}

  \item The C language imposes a constraint upon programmers
    who want to pass a multidimensional array to a function.
    What is the constraint? How did the design of the Java
    programming language eliminate that constraint for 
    programmers who use that language?


\begin{answer}
   In C, programmers are required to pass the length of an array to a method. Specifying the length of an array (like in C) is unnecessary because Java arrays store a pointer to an address containing their length.
    \end{end}

  \item An activation record contains a return
    address, a dynamic link, parameters, and
    local variables.
  \begin{enumerate}
    \item To what does the return address point?

    \item To what does the dynamic link point?

      \begin{answer}
      The return address usually consists of a pointer to the instruction following the call in the code segment of the calling program unit.
      \end{answer}

    \item To what does the dynamic link point?
      \begin{answer}
        The dynamic link points to the base of the activation record instance of the caller.
        \end{answer}

    \end{enumerate}

  \item The stack will contain multiple activation
    records for a single subprogram under what
    circumstances?

  \item How (or why?) does the LIFO protocol apply to
    calls to and returns from subprograms?

  \begin{answer}
    In a recursive program there can be multiple activation records (although they will be incomplete) for a single subprogram.
    \end{answer}

  \item How (or why?) does the LIFO protocol apply to
    calls to and returns from subprograms?
  
  \begin{answer}
    LIFO allows for subprograms to be nested within one another and for subprograms to be used as parameters within other subprograms. (417)
    \end{answer}

  \item Which important development in computer architecture
    has changed the way that the stack is used in some
    systems for facilitating calls to and returns from
    subprograms?

  \item A dynamic chain contains a history of what?

  \item Which two numbers are needed to compute
    the address of a local variable in a subprogram?

  \item How does a Ruby module differ from a class?

\begin{answer}
    RISC (reduced instruction set computing) machines have parameters passed in registers in their compilers because RISC machines have more registers than CISC (complex instruction set computing) machines. Chapter 10 assumes parameters are passed in the stack though, as they had been in CISC machines.
    \end{answer}

  \item A dynamic chain contains a history of what?

\begin{answer}
    Dynamic chain represents the dynamic history of how execution got to its current position, which is always in the subprogram code whose activation record instance is on top of the stack.
    \end{answer}
 

  \item Which two numbers are needed to compute
    the address of a local variable in a subprogram?

\begin{answer}
    To compute the address of a local variable you need the (chain\_offset, 
local\_offset) pair.
    \end{answer}


  \item How does a Ruby module differ from a class?

    \begin{answer}
    Modules are unlike classes in that they cannot be instantiated or subclassed and do not define variables. Methods that are defined in a module include the  module’s name in their names.
    \end{answer}

  \item Memory for variables can be allocated on the heap
    and on the stack. In which place or places is memory
    allocated for objects in C++? in Java?

  \item What problems were solved by the addition
    of genericity to Java?

  \item What is the purpose of the static chain?

  \item What is a singleton?

  \item What are the two parts of the definition 
    of an abstract data type?

   \begin{answer}
    In C++, variables can be allocated to the heap either by making them “static” or by allocating memory with the keyword “new”. However, variables which are initialized during the execution of a function are allocated to the stack. Java behaves the same way. (216-217)
    \end{answer}

  \item What problems were solved by the addition
    of genericity to Java?

   \begin{answer}
    Generics allow a type or method to operate on objects of various types while providing compile-time type safety.
    \end{answer}

  \item What is the purpose of the static chain?

    \begin{answer}
    The static chain is a path of pointers which go from each function to its parent. They allow child subprograms to use variables which are local to their parent, grandparent, or farther up, without needing to copy those variables to the call stack.
    \end{answer}
  \item What is a singleton?

    \begin{answer}
    A singleton is a class which provides a global access point to a single instance. This is useful for tasks which only need one point of access, like a file system.
    \end{answer}

  \item What are the two parts of the definition 
    of an abstract data type?

   \begin{answer}
    1. A type definition which allows program units to declare variables of the type but hides the representation of objects of the type. 2. A set of operations for manipulating objects of the type.
    \end{answer}

  \end{enumerate}


