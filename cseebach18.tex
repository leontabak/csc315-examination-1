
\chapter{Cameron Seebach}

\begin{enumerate}
  \item Most programming languages require the use of brackets to
    enclose the index in a reference to an element of an array.
  \begin{enumerate}
    \item Identify a language the requires the use of parentheses
      to enclose the index in a reference to an element of an array.
    \item Why did the designers of the language choose parentheses
      rather than brackets?
    \end{enumerate}

  \begin{answer}

  \begin{enumerate}
    \item The original FORTRAN used parentheses to to enclose an array
      reference.
    \item At the time FORTRAN was first introduced, keypunch machines did not
      include the square bracket keys.
    \end{enumerate}

    \end{answer}

  \item What is the relationship between a lexeme and a token?

  \begin{answer}
    Lexemes and tokens are both produced by the lexer. A lexeme is a string that
    makes up some symbol in the language, while a token describes a whole
    category of lexemes.

    For example, an identifier lexeme is commonly a string of characters a-z and
    underscore. Regardless of what characters are chosen, the token is still
    identifier.
    \end{answer}

  \item
  \begin{enumerate}
    \item What kind of symbols are found at the internal nodes of a
      parse tree?
    \item What kind of symbols are found at the leaves of a parse tree?
    \end{enumerate}

  \begin{answer}

  \begin{enumerate}
    \item Non-terminal symbols are found at the internal nodes of a parse tree.
    \item Terminal symbols are found at the leaves of a parse tree. All the
      non-terminals have been replaced at the leaves of the tree.
    \end{enumerate}

    \end{answer}


  \item One of the most significant contributions from the developers
    of ALGOL 60 also limited the success of that language. What was
    that contribution?

  \begin{answer}

    Backus-Naur form described the syntax of ALGOL 60 in a way that was new
    and unfamiliar at the time.

    \end{answer}

  \item What problem were the creators of Common LISP trying to solve?

  \begin{answer}

    The creators of Common LISP sought to combine several different dialects of
    Lisp, including Scheme, into one language.

    \end{answer}

  \item What is an ambiguous context free grammar?

  \begin{answer}

    It is a context free grammar that can describe more than one distinct parse
    tree.

    \end{answer}

  \item Contrast the complexity of algorithms that can parse strings
    that conform to the most general kinds of context free grammars
    and the complexity of the algorithms that can parse strings that
    conform to the grammars of programming languages?

  \begin{answer}

    Algorithms that parse the most general CFGs have a time complexity of
    O(n**3). For a programming language grammar, algorithms exist that run in
    O(n) time.

    \end{answer}

  \item Java represents characters with Unicode. It is the first
    widely used programming language with this feature. What is the
    significance of this feature?

  \begin{answer}

    It is easier to write programs intended for an international audience in
    Java. Less special work is required to support character sets outside of
    ASCII.

    \end{answer}

  \item How does the binary coded decimal type differ from the
    floating point type?

  \begin{answer}

    Binary coded decimal typically allows for a fixed number of digits before
    the decimal point, and a fixed number of digits afterwards. By contrast,
    floating point types have a number of digits of precision which is not
    related to the location of the decimal point.

    Also, binary coded decimal is typically made up of groups of 4 bits (0-F),
    each four bits representing a digit (0-9). As a result, a binary coded
    decimal number takes more bits than a floating point number to store the
    same number.

    \end{answer}

  \item Identify a user-defined ordinal type in the Java programming
    language.

  \begin{answer}

    The user-defined ordinal type in Java is the Enum.

    \end{answer}

  \item Mathematicians and programmers might have different ideas
    about the precedence of Boolean operators. Explain.

  \begin{answer}

    In mathematics, AND and OR have the same precendence, but in the C language
    AND has a higher precedence.

    \end{answer}

  \item Programmers should use \verb+===+ rather than \verb+==+ to
    test the equality of the values of two expressions in JavaScript. Why?

  \begin{answer}

    Equality of strings and other objects is the best reason to use one over the
    other. \verb+===+ compares the content of two objects or strings,
    while \verb+==+ compares only the identity.

    \end{answer}

  \item Describe a hazard of allowing short-circuited evaluation
    of expressions and side effects in expressions at the same time.

  \begin{answer}

    If the expression with a side effect is placed second in a short-circuit
    operation, it may not always be applied if the operation short-circuits.

    This can lead to a situation where a programmer thinks that a mutating
    change has occurred, when in fact it has not.

    \end{answer}

  \item Briefly describe the three steps in the mark-sweep algorithm
    for garbage collection.

  \begin{answer}

    \begin{enumerate}
      \item Traverse the tree of objects reachable from the `root set', marking
        each as in-use.
      \item Scan all memory, freeing objects which are not in-use.
      \item Reset the in-use flag on each object still extant.

      \end{enumerate}

    \end{answer}

  \item What led Yukihiro Matsumoto to create the Ruby programming language?

  \begin{answer}

    Matsumoto wanted an easy, truly object-oriented scripting language, but
    didn't like the options available at that time in 1993.

    \end{answer}

  \item What did Microsoft aim to achieve with its development of the
    C\# language?

  \begin{answer}

    Microsoft aimed to have a flagship language similar to C and C++ for use
    with its new Common Language Runtime. Portability and general-purposeness
    were other important design goals.

    \end{answer}

  \end{enumerate}

\section{More questions for discussion and review.}

\begin{enumerate}
  \item The design of which machine influenced the design
    of the control statements in FORTRAN?

  \begin{answer}
    The IBM 704 mainframe computer's assembly language influenced the control 
    statements in the first version of FORTRAN.
    \end{answer}

  \item How many different kinds of control statements
    must the designer of a programming language include
    in a language?

  \begin{answer}
    Technically, just one, a ``selectable goto''. However, designers usually 
    include at least one selection statement and one iteration statement.
    \end{answer}

  \item What is the one question that applies in the
    design of all statements that allow selection or
    iteration?

  \begin{answer}
    Should the control structure have multiple statements? In other words, 
    should execution always begin with the first statement in the structure?
    \end{answer}

  \item What is an advantage of requiring that
    the \textbf{then} and \textbf{else} clauses of
    an \textbf{if} statement be compound statements?

  \begin{answer}
    Readability is increased. It is easier to distinguish which statements are
    in an if statement body when only compound statements are allowed.
    \end{answer}

  \item How does the \textbf{switch} statement in C\#
    differ from the \textbf{switch} statement in Java?

  \begin{answer}
    C\# allows only one of the paths to be executed, the first matching. Java 
    requires a \textbf{break} statement to be added to each path for this 
    behaviour.
    \end{answer}

  \item Distinguish between 2 statements in Ruby
    that correspond to Java's \textbf{switch} statement.

  \begin{answer}
    One of these forms is similar to a series of nested if statements, and 
    evaluates to a value. The other is similar to the switch statement in Java
    \end{answer}

  \item Features of a programming language sometimes persist
    longer than a feature of computing hardware that inspired
    and supported that part of the language's design.
    Similarly, features of hardware sometimes persist longer
    than some parts of a language's design that were created
    to take advantage of that feature in hardware.

    Give examples.

  \begin{answer}
    The \textbf{register} keyword in C was intended as a hint to the compiler
    that a variable should be kept on a CPU register instead of in memory. 
    Compiler technology has improved so much that the compiler and optimizer 
    make better decisions about what to put in registers than humans can, and
    this keyword has thus fallen into disuse and deprecation.

    In the x86 assembly language, there is limited support for Binary Coded 
    Decimal arithmetic. Modern programming languages do not have built-in 
    Binary Coded Decimal types, and yet the BCD opcodes in the x86 architecture have 
    been retained for backwards compatibility.
    \end{answer}

  \item Who most famously warned of the dangers of using the
    \textbf{goto} statement? What did Donald Knuth have to
    say about the use of the \textbf{goto} statement?

  \begin{answer}
    Edsger Dijkstra's famous letter \textit{Go To Statement Considered Harmful} 
    warned against the use of \textbf{goto}. Donald Knuth said it was usually a 
    bad idea to use \textbf{goto}, but that there were a few exceptional cases
    where it improved the readability of a program.
    \end{answer}

  \item Describes Ada's \textbf{for} loop. Are there some
    kinds of iteration that might be easier in Ada than
    in Java? Easier in Java than in Ada?

  \begin{answer}
    Ada's for loop is less flexible than the for loop in Java, but as a result 
    is more constrained and easier to read. A reverse iteration over a discrete 
    range is mildly easier in Ada than Java, as Ada has the built-in reverse 
    keyword for these situations. Java has easier iteration over Iterable 
    objects, because Ada requires that the iterable be over a discrete range and 
    not from an object.
    \end{answer}

  \item What does it mean to say that the guarded commands
    of Ada are non-deterministic?

  \begin{answer}
    Any of the guards that evaluate to true may have their expression chosen for 
    execution, at random. This is part of Ada's concurrency support.
    \end{answer}

  \item The header files in a C program contain function
    prototypes. What is a function prototype?

  \begin{answer}
    A function prototype contains the name, return type, and arguments of a function.
    \end{answer}

  \item Every method in a Ruby program belongs to a class.
    A programmer can place a definition of a method inside
    the definition of a class or outside of the definition
    of any class that the programmer writes. To which class
    does the method belong in the second case?

  \begin{answer}
    It belongs to the Object class.
    \end{answer}

  \item Distinguish between positional and keyword parameters.

  \begin{answer}
    Positional parameters must be passed in a distinct order, while keyword 
    parameters can be passed in any order and may sometimes even not be passed,
    using a default value.
    \end{answer}

  \item Ruby blocks are closures. What does that mean?

  \begin{answer}
    A Ruby block is an anonymous subprogram passed to a method. It can reference
    its enclosing environment.
    \end{answer}

  \item What is a pure function?

  \begin{answer}
    A pure function is a function without side effects, meaning it does not modify
    the global state of the computer.
    \end{answer}

  \item Some languages give programmers means to define
    both functions and procedures. Java does not. Is that
    a serious limitation?

  \begin{answer}
    This is not a serious limitation, because one can declare a function with return
    type void in Java, mimicking a procedure.
    \end{answer}

  \item Declarations of formal parameters in an Ada procedure
    can include, in addition to the names and types of the
    parameters, reserved words that do not appear in declarations
    in Java programs. 
    What is the purpose of those reserved words?

  \begin{answer}
    Those words are \textbf{in}, \textbf{out}, and \textbf{inout}. They specify the 
    modifiability of the parameters they describe. In parameters cannot be modified 
    by the function. Out parameters cannot be read. Inout parameters can be both 
    modified and read.
    \end{answer}
 
  \item The C language imposes a constraint upon programmers
    who want to pass a multidimensional array to a function.
    What is the constraint? How did the design of the Java
    programming language eliminate that constraint for 
    programmers who use that language?

  \begin{answer}
    A programmer must pass the number of columns in a two-dimensional array as part 
    of the type signature, so that the compiler can build a mapping function. Java 
    arrays carry with them a length attribute, eliminating this requirement.
    \end{answer}

  \item An activation record contains a return
    address, a dynamic link, parameters, and
    local variables.
  \begin{enumerate}
    \item To what does the return address point?
    \item To what does the dynamic link point?
    \end{enumerate}

  \begin{answer}
    The return address points to the next instruction to be executed in the calling 
    code.
    The dynamic link points to the base of the activation record instance of the 
    caller.
    \end{answer}

  \item The stack will contain multiple activation
    records for a single subprogram under what
    circumstances?

  \begin{answer}
    The stack behaves like this when recursive subprograms are allowed.
    \end{answer}

  \item How (or why?) does the LIFO protocol apply to
    calls to and returns from subprograms?

  \begin{answer}
    Subprogram activation records are added to the stack, and then the top is popped
    off as subprograms finish. The last run subprogram is always at the top.
    \end{answer}

  \item Which important development in computer architecture
    has changed the way that the stack is used in some
    systems for facilitating calls to and returns from
    subprograms?

  \begin{answer}
    The introduction of the call stack allowed for recursive subprograms.
    \end{answer}

  \item A dynamic chain contains a history of what?

  \begin{answer}
    Which subprograms have been called and in what order to get to the current 
    subprogram.
    \end{answer}

  \item Which two numbers are needed to compute
    the address of a local variable in a subprogram?

  \begin{answer}
    You need the EP - the Environment Pointer - plus an offset.
    \end{answer}

  \item How does a Ruby module differ from a class?

  \begin{answer}
    A module cannot be instantiated - it is just a collection of methods and objects.
    \end{answer}

  \item Memory for variables can be allocated on the heap
    and on the stack. In which place or places is memory
    allocated for objects in C++? in Java?

  \begin{answer}
    In C++, local variables are on the stack, and variables that outlive a single
    function live on the heap and are created by \textbf{new}.
    In Java all objects live on the heap and are garbage collected.
    \end{answer}

  \item What problems were solved by the addition
    of genericity to Java?

  \begin{answer}
    Casts had to be done back and forth into and out of a data structure prior to 
    generics.
    \end{answer}

  \item What is the purpose of the static chain?

  \begin{answer}
    The static chain allows the access of non-local variables.
    \end{answer}

  \item What is a singleton?

  \begin{answer}
    A singleton is a class that is instantiated only once.
    \end{answer}

  \item What are the two parts of the definition 
    of an abstract data type?

  \begin{answer}
    A type definition and a set of operations defined on that type.
    \end{answer}

  \end{enumerate}


