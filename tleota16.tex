
\chapter{Tuli Leota}

\begin{enumerate}
  \item Most programming languages require the use of brackets to
    enclose the index in a reference to an element of an array.
  \begin{enumerate}
    \item Identify a language the requires the use of parentheses
      to enclose the index in a reference to an element of an array.
    \item Why did the designers of the language choose parentheses
      rather than brackets?
    \end{enumerate}

  \begin{answer}

  \begin{enumerate}
    \item Write your answer here.
    \item Write your answer here.
    \end{enumerate}

    \end{answer}
    
  \item What is the relationship between a lexeme and a token?

  \begin{answer}

    Write your answer here.

    \end{answer}

  \item
  \begin{enumerate}
    \item What kind of symbols are found at the internal nodes of a
      parse tree?
    \item What kind of symbols are found at the leaves of a parse tree?
    \end{enumerate}

  \begin{answer}

  \begin{enumerate}
    \item Write your answer here.
    \item Write your answer here.
    \end{enumerate}

    \end{answer}


  \item One of the most significant contributions from the developers
    of ALGOL 60 also limited the success of that language. What was
    that contribution?

  \begin{answer}

    Write your answer here.

    \end{answer}

  \item What problem were the creators of Common LISP trying to solve?

  \begin{answer}

    Write your answer here.

    \end{answer}

  \item What is an ambiguous context free grammar?

  \begin{answer}

    Write your answer here.

    \end{answer}

  \item Contrast the complexity of algorithms that can parse strings
    that conform to the most general kinds of context free grammars
    and the complexity of the algorithms that can parse strings that
    conform to the grammars of programming languages?

  \begin{answer}

    Write your answer here.

    \end{answer}

  \item Java represents characters with Unicode. It is the first
    widely used programming language with this feature. What is the
    significance of this feature?

  \begin{answer}

    Write your answer here.

    \end{answer}

  \item How does the binary coded decimal type differ from the
    floating point type?

  \begin{answer}

    Write your answer here.

    \end{answer}

  \item Identify a user-defined ordinal type in the Java programming
    language.

  \begin{answer}

    Write your answer here.

    \end{answer}

  \item Mathematicians and programmers might have different ideas
    about the precedence of Boolean operators. Explain.

  \begin{answer}

    Write your answer here.

    \end{answer}

  \item Programmers should use \verb+===+ rather than \verb+==+ to
    test the equality of the values of two expressions in JavaScript. Why?

  \begin{answer}

    Write your answer here.

    \end{answer}

  \item Describe a hazard of allowing short-circuited evaluation
    of expressions and side effects in expressions at the same time.

  \begin{answer}

    Write your answer here.

    \end{answer}

  \item Briefly describe the three steps in the mark-sweep algorithm
    for garbage collection.

  \begin{answer}

    Write your answer here.

    \end{answer}

  \item What led Yukihiro Matsumoto to create the Ruby programming language?

  \begin{answer}

    Write your answer here.

    \end{answer}

  \item What did Microsoft aim to achieve with its development of the
    C\# language?

  \begin{answer}

    Write your answer here.

    \end{answer}

  \end{enumerate}



\section{More questions for discussion and review.}

\begin{enumerate}
  \item The design of which machine influenced the design
    of the control statements in FORTRAN?

  \begin{answer}

    The IBM 704 influenced the design of the control statements in FORTRAN.

    \end{answer}

  \item How many different kinds of control statements
    must the designer of a programming language include
    in a language?

  \begin{answer}

    While it is possible to use only one goto, the minimum number of conrol
statements in a language that doesn't use goto is two.  The first for choosing
between control flow paths, and the other for logically controlled iterations.

    \end{answer}

  \item What is the one question that applies in the
    design of all statements that allow selection or
    iteration?

 \begin{answer}

    Should the control structure have mulitiple entries?

    \end{answer}

  \item What is an advantage of requiring that
    the \textbf{then} and \textbf{else} clauses of
    an \textbf{if} statement be compound statements?

 \begin{answer}

    Requiring compound statements helps increase the readability and writability for
 programmers when using nested selector statements, that otherwise can get very
 messy and complicated.

    \end{answer}

  \item How does the \textbf{switch} statement in C\#
    differ from the \textbf{switch} statement in Java?

 \begin{answer}

    In Java, the switch statment doesn't allow case expressions anywhere except
the top level of the body of the switch.  C allows the execution of more than one
segment.  In C the control expressions as well as case statements can be strings.

    \end{answer}

  \item Distinguish between 2 statements in Ruby
    that correspond to Java's \textbf{switch} statement.
    
    \begin{answer}

    Case expressions are the Ruby equivalent of Java's \textbf{switch} statement. One is semantically similar to nested if statements with case - when - then. The other is with boolean expressions being evaluated one at a time from top to bottom. The value of this case expression is equivalent to the value of the \textbf{then} statement corresponding to the first \textit{true} \textbf{when} statement.

    \end{answer}
    

  \item Features of a programming language sometimes persist
    longer than a feature of computing hardware that inspired
    and supported that part of the language's design.
    Similarly, features of hardware sometimes persist longer
    than some parts of a language's design that were created
    to take advantage of that feature in hardware.

    Give examples.

 \begin{answer}

The IBM 704 influenced the design of control statements which are still used today, and prompted the development of Fortran.  Von Neumann architecture.  The Algol standard used several different syntaxes which, among other things, allowed Europeans to use a comma to denote a decimal point, while Americans could continue to use a period.

    \end{answer}

  \item Who most famously warned of the dangers of using the
    \textbf{goto} statement? What did Donald Knuth have to
    say about the use of the \textbf{goto} statement?

 \begin{answer}

 Edsger Dijkstra noted that the goto statement was too primitive and could
easily make a mess of one's program.  Donald Knuth argued that there were
occasions when the efficiency of the goto outweighed its harm to readability.

    \end{answer}


  \item Describes Ada's \textbf{for} loop. Are there some
    kinds of iteration that might be easier in Ada than
    in Java? Easier in Java than in Ada?

  \begin{answer}

    The counter variable can't be changed in the body.  You gain flexibility by
allowing you to edit any part of the loop.

    \end{answer}

  \item What does it mean to say that the guarded commands
    of Ada are non-deterministic?

 \begin{answer}

Guarded commands in Ada are nondeterministically chosen for execution when more than
 one of the statements are evaluated to true. This means that if there are three
 guarded statements and two of the three evaluate to true, then each time the
 program will use one of the two statements. It will not always use the one that
 appears first nor the one that appears last, but rather it will choose between them
 non-deterministically or randomly.
    \end{answer}

  \item The header files in a C program contain function
    prototypes. What is a function prototype?

\begin{answer}

A function prototype is a function declaration that gives the function's name and
 type signature, but does not specify the function body. It is also referred to as
 a function interface at times. In other languages these are uncommon because
 subprograms do not need declarations since they do no need to be defined before
 they are called. 
    \end{answer}

  \item Every method in a Ruby program belongs to a class.
    A programmer can place a definition of a method inside
    the definition of a class or outside of the definition
    of any class that the programmer writes. To which class
    does the method belong in the second case?

\begin{answer}

If a method is defined outside of the definition of any class that the programmer
 writes then the method belongs to the root object, \textbf{Object}.
    \end{answer}

  \item Distinguish between positional and keyword parameters.

\begin{answer}
Positional parameters are bound based on the order in which they are given to the
 function. Keyword parameters are when the name of the formal parameter to which an
 actual parameter is to be bound is specified with the actual parameter in a call.
 This allows them to be input in any order.Positional parameters are bound based on
 the order in which they are given to the function. Keyword parameters are when the
 name of the formal parameter to which an actual parameter is to be bound is
 specified with the actual parameter in a call. This allows them to be input in any
 order.

    \end{answer}

  \item Ruby blocks are closures. What does that mean?

\begin{answer}

A closure is an anonymous function which can be passed as a parameter. Also, Ruby
 doesn't nest scope, so variables defined in a method are not accessible outside of
 the method (say, in the class which defines them).

    \end{answer}

  \item What is a pure function?

\begin{answer}


A pure function is a function where the return value is only determined by its
input values, without observable side effects.

    \end{answer}

  \item Some languages give programmers means to define
    both functions and procedures. Java doe not. Is that
    a serious limitation?

\begin{answer}

It does not seem to be a serious limitation, it just means that performing procedures
 in Java is more verbose because it requires that the programmer instantiate the
 class with any needed methods.


    \end{answer}

  \item Declarations of formal parameters in an Ada procedure
    can include, in addition to the names and types of the
    parameters, reserved words that do not appear in declarations
    in Java programs. 
    What is the purpose of those reserved words?

\begin{answer}

Ada allows the programmer to specify in mode, out mode, and inout mode for each
 formal parameter.  This means that they can receive data from the actual parameter,
 they can send data to the actual parameter, or they can do both.

    \end{answer}
 
  \item The C language imposes a constraint upon programmers
    who want to pass a multidimensional array to a function.
    What is the constraint? How did the design of the Java
    programming language eliminate that constraint for 
    programmers who use that language?

\begin{answer}

    In C, programmers are required to pass the length of an array to a method.
 Specifying the length of an array (like in C) is unnecessary because Java arrays
 store a pointer to an address containing their length.

    \end{answer}

  \item An activation record contains a return
    address, a dynamic link, parameters, and
    local variables.
  \begin{enumerate}
    \item To what does the return address point?
    \item To what does the dynamic link point?

\begin{answer}

    The return address consists of a pointer to the instruction following the call
in the code segment of the calling program unit.  The dynamic link points to the base
of the activation record instance of the caller.

    \end{answer}
    \end{enumerate}

  \item The stack will contain multiple activation
    records for a single subprogram under what
    circumstances?

\begin{answer}

    In a recursive program although they will be incomplete.

    \end{answer}

  \item How (or why?) does the LIFO protocol apply to
    calls to and returns from subprograms?

\begin{answer}

    LIFO allows for subprograms to be nested within one another and for subprograms
to be used as parameters within other subprograms.

    \end{answer}

  \item Which important development in computer architecture
    has changed the way that the stack is used in some
    systems for facilitating calls to and returns from
    subprograms?

 \begin{answer}

    RISC (reduced instruction set computing) machines have parameters passed in
 registers in their compilers because RISC machines have more registers than CISC
 (complex instruction set computing) machines. Chapter 10 assumes parameters are
 passed in the stack though, as they had been in CISC machines.

    \end{answer}

  \item A dynamic chain contains a history of what?

 \begin{answer}

    A dynamic chain contains history of how execution got to its current position,
which is always in the subprogram code whose activation record instance is on top
of the stack.

    \end{answer}

  \item Which two numbers are needed to compute
    the address of a local variable in a subprogram?

 \begin{answer}

    You need the (chain_offset, local_offset) pair.

    \end{answer}

  \item How does a Ruby module differ from a class?

 \begin{answer}

    Modules differ from classes because they can't be instantiated or subclassed
and don't define variables.  Methods that are defined in a module include the
modules name in their names.

    \end{answer}

  \item Memory for variables can be allocated on the heap
    and on the stack. In which place or places is memory
    allocated for objects in C++? in Java?

 \begin{answer}

    In C++ variables can be allocated on the heap either by making them static
or by allocating memory with the keyword ``new''.  Java is the same.

    \end{answer}

  \item What problems were solved by the addition
    of genericity to Java?

 \begin{answer}

    Genericity allow a type or method to operate on objects of various types
while providing compile-time type safety.

    \end{answer}

  \item What is the purpose of the static chain?

 \begin{answer}

    The static chain is a path of pointers which go from each function to its
parent. They allow child subprograms to use variables which are local to their
parent, grandparent, or further up, without needing to copy those variables to the
call stack.

    \end{answer}

  \item What is a singleton?

 \begin{answer}

    A singleton is a class that provides a global access point to a single instance.

    \end{answer}

  \item What are the two parts of the definition 
    of an abstract data type?

 \begin{answer}

    1.) A type definition which allows program units to declare variables of the 
type, but hides the representation of it.
    2.) A set of operations for manipulating objects of the type.

    \end{answer}

  \end{enumerate}


