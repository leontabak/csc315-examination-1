
\chapter{Aeint Thet Ngon}

\begin{enumerate}
  \item Most programming languages require the use of brackets to
    enclose the index in a reference to an element of an array.
  \begin{enumerate}
    \item Identify a language the requires the use of parentheses
      to enclose the index in a reference to an element of an array.
    \item Why did the designers of the language choose parentheses
      rather than brackets?
    \end{enumerate}

  \begin{answer}

  \begin{enumerate}
    \item Ada
    \item Because the designers wanted uniformity between array references and function calls in expressions in spite of potential readability problems. 
    \end{enumerate}

    \end{answer}
    
  \item What is the relationship between a lexeme and a token?

  \begin{answer}

    A lexeme is a formal descriptions of a syntax of progamming languages and most of the times do not include descriptions of the lowest-level syntactic units. Each lexeme group is represented by a name or a token. So a token of a language is a category of its lexemes.

    \end{answer}

  \item
  \begin{enumerate}
    \item What kind of symbols are found at the internal nodes of a
      parse tree?
    \item What kind of symbols are found at the leaves of a parse tree?
    \end{enumerate}

  \begin{answer}

  \begin{enumerate}
    \item Non-terminal cateogires of the grammar are found at the internal nodes of a parse tree.
    \item Leaf nodes are labelled by terminal categories.
    \end{enumerate}

    \end{answer}


  \item One of the most significant contributions from the developers
    of ALGOL 60 also limited the success of that language. What was
    that contribution?

  \begin{answer}

    BNF, one of the most important contributions to computer science, is considered a simple and elegant means of syntax description but in 1960 it seemed strange and complicated and was a factor in its lack of acceptance.

    \end{answer}

  \item What problem were the creators of Common LISP trying to solve?

  \begin{answer}

    During the 1970s and early 198s, due to the usage of diverse dialects of LISP, there was a problem of lack of portability among the programs writting using different dialects. To solve the problem, Common Lisp was created by combining the features of different dialects of LISP.

    \end{answer}

  \item What is an ambiguous context free grammar?

  \begin{answer}

    A grammar that genereates a sentential form for which there are two or more distinct parse trees is said to be ambiguous.

    \end{answer}

  \item Contrast the complexity of algorithms that can parse strings
    that conform to the most general kinds of context free grammars
    and the complexity of the algorithms that can parse strings that
    conform to the grammars of programming languages?

  \begin{answer}

    Parsing algorithms that work for any unambiguous grammar are complicated and inefficient, the amount of time they take is on the order of the cube of the lenght of the string to be parsed. So generality is traded for efficiency. Faster alogrithms have been found that work only for a subset of the set of all possible grammars and the time they take is linearly related to the length of the string to be parsed.

    \end{answer}

  \item Java represents characters with Unicode. It is the first
    widely used programming language with this feature. What is the
    significance of this feature?

  \begin{answer}

    This feature includes the characters from most of the world's natural languages. It was developed because of globalization of business and the need for computers to communicate with other computers around the world.

    \end{answer}

  \item How does the binary coded decimal type differ from the
    floating point type?

  \begin{answer}

    Floating point data types model real numbers but the representation are only apprximations for many real values. Decimal types have the advantage of being able to precisely store decimal values, which cannot be done with floating point.

    \end{answer}

  \item Identify a user-defined ordinal type in the Java programming
    language.

  \begin{answer}

    Enumeration Types

    \end{answer}

  \item Mathematicians and programmers might have different ideas
    about the precedence of Boolean operators. Explain.

  \begin{answer}

    In mathematics, Boolean algebras have equal precedence, however, C-based languages assign higher precedence to AND than OR. This might have resulted from the baseless correlation of mulitplcation with AND and of additon with OR, which would then naturally assign higher precedence to AND.

    \end{answer}

  \item Programmers should use \verb+===+ rather than \verb+==+ to
    test the equality of the values of two expressions in JavaScript. Why?

  \begin{answer}

    Because whena  string and a number are the operands of a relational operator, the string is coerced to a number but when \verb+===+ is used no coercion is done on the operands of this operator.

    \end{answer}

  \item Describe a hazard of allowing short-circuited evaluation
    of expressions and side effects in expressions at the same time.

  \begin{answer}

    A language that provides short-circuit evaluations of Boolean expressions and also has side effects in expressions allows subltel errors to occr. Suppose that short-circuit evaluation is used on an expression and part of the expression that contains a side effect is not evaluated; then the side effect will occur only in complete evaluations of the whole expression. If a program correctness depends on the side effect, short-circuit evaluation can result in a serious error.

    \end{answer}

  \item Briefly describe the three steps in the mark-sweep algorithm
    for garbage collection.

  \begin{answer}

    \begin{itemize}
      \item{All cells in the heap have their indicators set to indicate they are garbage}
      \item{Marking phase-Every pointer in the program is traced into the heap and all reachable cells are marked as not being garbage}
      \item{Sweep phase-all cells in the heap that have not been specifially marked as still being used are returned to the list of available space}
    \end{itemize}

    \end{answer}

  \item What led Yukihiro Matsumoto to create the Ruby programming language?

  \begin{answer}

    Yukihiro Matsumoto's motivation was due to his dissatisfaction with Perl and Python, which support object-oriented programming but neither is puer object oriented langauge.

    \end{answer}

  \item What did Microsoft aim to achieve with its development of the
    C\# language?

  \begin{answer}

    The purose of C\# is to provide a language for componenet based software development. In this environment, components from a variety of languages can be easily combined to form systems. 

    \end{answer}

  \end{enumerate}



\section{More questions for discussion and review.}

\begin{enumerate}
  \item The design of which machine influenced the design
    of the control statements in FORTRAN?
    
    \begin{answer}
    
    IBM 704 influenced the design of the control statements in FORTRAN.
    
    \end{answer}

  \item How many different kinds of control statements
    must the designer of a programming language include
    in a language?
    
    \begin{answer}
    
     One of the primary conclusions of these efforts was that, although a single control statement (a selectable goto) is minimally sufficient, a language that is designed not to include a goto needs only a small number of different control statements.
     
    \end{answer}

  \item What is the one question that applies in the
    design of all statements that allow selection or
    iteration?
    
    \begin{answer}
     Should multiple entries be permitted?
    \end{answer}

  \item What is an advantage of requiring that
    the \textbf{then} and \textbf{else} clauses of
    an \textbf{if} statement be compound statements?
    
    \begin{answer}
    
    Requiring compound statements helps increase the readability and writability for programmers when using nested selector statements, that otherwise can get very messy and complicated.
    
    \end{answer}

  \item How does the \textbf{switch} statement in C\#
    differ from the \textbf{switch} statement in Java?
    
    \begin{answer}
    The C\# switch statement differs from that of its C-based predecessors
	in two ways. First, C\# has a static semantics rule that disallows the 			implicit execution of more than one segment. The rule is that every 			selectable segment must end with an explicit unconditional branch statement: 	 either a break, which transfers control out of the switch statement, or a 		goto, which can transfer control to one of the selectable segments (or 			virtually anywhere else). The other way C\#’s switch differs from that of 		its predecessors is that the control expression and the case statements can 	be strings in C\#.

    \end{answer}

  \item Distinguish between 2 statements in Ruby
    that correspond to Java's \textbf{switch} statement.
    
    \begin{answer}
    
    Ruby's case expression is similar to the switch statement. And \textbf{when} in Ruby's case expression is used to evaluate as \textbf{if} does in Java's switch statement and \textbf{then} statement is executed.
    
    \end{answer}

  \item Features of a programming language sometimes persist
    longer than a feature of computing hardware that inspired
    and supported that part of the language's design.
    Similarly, features of hardware sometimes persist longer
    than some parts of a language's design that were created
    to take advantage of that feature in hardware.

    Give examples.
    
    \begin{answer}
    
    The IBM 704 influenced the design of control statements
    
    \end{answer}

  \item Who most famously warned of the dangers of using the
    \textbf{goto} statement? What did Donald Knuth have to
    say about the use of the \textbf{goto} statement?
    
    \begin{answer}
    
    Donald Knuth argues that well thought out, rational, controlled use of goto is not entirely harmful but uncontrolled and thoughtless use of GoTo is probably a bad thing.
    
    \end{answer}

  \item Describes Ada's \textbf{for} loop. Are there some
    kinds of iteration that might be easier in Ada than
    in Java? Easier in Java than in Ada?
    
    \begin{answer}
    
    Ada’s for loop can use any ordinal type variable for its counter. Arrays with ordinal type subscripts can be conveniently processed
    
    \end{answer}
    
  \item What does it mean to say that the guarded commands
    of Ada are non-deterministic?
    
    \begin{answer}
    
    A guarded command is a statement, or list of statements, that is "guarded" by a boolean expression.The semantics is that all the expressions are evaluated simultaneously, if one of the expressions is true, then the statement associated with it is performed. If more that one is true, then one statement is picked nondeterministically. Finally, if all of the expressions evaluate to false, the program exits the loop.
    
    \end{answer}

  \item The header files in a C program contain function
    prototypes. What is a function prototype?
    
    \begin{answer}
    
     A function prototype or function interface is a declaration of a function that specifies the function's name and type signature (arity, parameter types, and return type), but omits the function body.

    \end{answer}

  \item Every method in a Ruby program belongs to a class.
    A programmer can place a definition of a method inside
    the definition of a class or outside of the definition
    of any class that the programmer writes. To which class
    does the method belong in the second case?
    
    \begin{answer}
     If a method is defined outside of the definition of any class that the programmer writes then the method belongs to the root object, \textbf{Object}
    \end{answer}

  \item Distinguish between positional and keyword parameters.
  
  \begin{answer}
  Positional parameters are bound based on the order in which they are given to the function. Keyword parameters are when the name of the formal parameter to which an actual parameter is to be bound is specified with the actual parameter in a call. This allows them to be input in any order
  \end{answer}

  \item Ruby blocks are closures. What does that mean?
  
  \begin{answer}
  
  A closure is an anonymous function which can be passed as a parameter. Also, Ruby doesn't nest scope, so variables defined in a class are not accessible in that class's methods.
  
  \end{answer}

  \item What is a pure function?
  
  \begin{answer}
  
  A pure function is a function where the return value is only determined by its input values, without observable side effects. This is how functions in math work: Math.cos(x) will, for the same value of x , always return the same result.
  
  \end{answer}

  \item Some languages give programmers means to define
    both functions and procedures. Java doe not. Is that
    a serious limitation?
    
    \begin{answer}
    
    It does not seem to be a serious limitation, it just means that performing procedures in Java is more verbose because it requires that the programmer instantiate the class with any needed methods.
    
    \end{answer}

  \item Declarations of formal parameters in an Ada procedure
    can include, in addition to the names and types of the
    parameters, reserved words that do not appear in declarations
    in Java programs. 
    What is the purpose of those reserved words?
    
    \begin{answer}
    
    Ada and Fortran 95+ allow the programmer to specify in mode, out mode, or inout mode on each formal parameter.A reserved word is a special word of a programming language that cannot be used as a name. As a language design choice, reserved words are better than keywords because the ability to redefine keywords can be confusing.

    \end{answer}
 
  \item The C language imposes a constraint upon programmers
    who want to pass a multidimensional array to a function.
    What is the constraint? How did the design of the Java
    programming language eliminate that constraint for 
    programmers who use that language?
    
    \begin{answer}
    
    In C, programmers are required to pass the length of an array to a method. Specifying the length of an array (like in C) is unnecessary because Java arrays store a pointer to an address containing their length.

    \end{answer}

  \item An activation record contains a return
    address, a dynamic link, parameters, and
    local variables.
  \begin{enumerate}
    \item To what does the return address point?
    \item To what does the dynamic link point?
    \end{enumerate}
    
    \begin{answer}
    
The return address usually consists of a pointer to the instruction following the call in the code segment of the calling program unit.  The dynamic link points to the base of the activation record instance of the caller.

    \end{answer}

  \item The stack will contain multiple activation
    records for a single subprogram under what
    circumstances?
    
    \begin{answer}
    In a recursive program there can be multiple activation records (although they will be incomplete) for a single subprogram.
    \end{answer}

  \item How (or why?) does the LIFO protocol apply to
    calls to and returns from subprograms?
    
    \begin{answer}
    LIFO allows for subprograms to be nested within one another and for subprograms to be used as parameters within other subprograms.
    \end{answer}

  \item Which important development in computer architecture
    has changed the way that the stack is used in some
    systems for facilitating calls to and returns from
    subprograms?
    
    \begin{answer}
    RISC (reduced instruction set computing) machines have parameters passed in registers in their compilers because RISC machines have more registers than CISC (complex instruction set computing) machines. Chapter 10 assumes parameters are passed in the stack though, as they had been in CISC machines.

    \end{answer}

  \item A dynamic chain contains a history of what?
  
  \begin{answer}
  Dynamic chain represents the dynamic history of how execution got to its current position, which is always in the subprogram code whose activation record instance is on top of the stack.
  \end{answer}

  \item Which two numbers are needed to compute
    the address of a local variable in a subprogram?
    
    \begin{answer}
    
    To compute the address of a local variable you need the (chain\textunderscore offset, local\textunderscore offset) pair.

    \end{answer}

  \item How does a Ruby module differ from a class?
  
  \begin{answer}
  
  Modules are unlike classes in that they cannot be instantiated or subclassed and do not define variables. Methods that are defined in a module include the  module’s name in their names
  
  \end{answer}

  \item Memory for variables can be allocated on the heap
    and on the stack. In which place or places is memory
    allocated for objects in C++? in Java?
    
    \begin{answer}
    In C++, variables can be explicitly fixed in the heap by using the keyword “static”. They can also be also be dynamically allocated to the heap with the new keyword.
    \end{answer}

  \item What problems were solved by the addition
    of genericity to Java?
    
    \begin{answer}
    
    Generics allow a type or method to operate on objects of various types while providing compile-time type safety.
    
    \end{answer}

  \item What is the purpose of the static chain?
  
  \begin{answer}
  
  Static chain points to the bottom of the activation record instance of an activation of the static parent. It is used for accesses to nonlocal variables. 
  
  \end{answer}

  \item What is a singleton?
  
  \begin{answer}
  
  A singleton is a class which provides a global access point to a single instance. This is useful for tasks which only need one point of access, like a file system.

  \end{answer}

  \item What are the two parts of the definition 
    of an abstract data type?
    
    \begin{answer}
    \begin{enumerate}
\item{A type definition which allows program units to declare variables of the type but hides the representation of objects of the type}
\item{A set of operations for manipulating objects of the type.}
	\end{enumerate}
    \end{answer}
\end{enumerate}


