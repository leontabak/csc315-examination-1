
\chapter{Sam Caldwell}

\begin{enumerate}
  \item Most programming languages require the use of brackets to
    enclose the index in a reference to an element of an array.
  \begin{enumerate}
    \item Identify a language the requires the use of parentheses
      to enclose the index in a reference to an element of an array.
    \item Why did the designers of the language choose parentheses
      rather than brackets?
    \end{enumerate}

  \begin{answer}

  \begin{enumerate}
    \item Ada
    \item The designers of Ada specifically chose parentheses to
      enclose subscripts so there would be uniformity between array
      references and function calls in expressions, in spite of
      potential readability problems
    \end{enumerate}

    \end{answer}
    
  \item What is the relationship between a lexeme and a token?

  \begin{answer}

    A token represents functional groups of lexemes, which are
    representations of the lowest level of syntactic units

    \end{answer}

  \item
  \begin{enumerate}
    \item What kind of symbols are found at the internal nodes of a
      parse tree?
    \item What kind of symbols are found at the leaves of a parse tree?
    \end{enumerate}

  \begin{answer}

  \begin{enumerate}
    \item internal nodes have non-terminal symbols
    \item leaf nodes have terminal symbols
    \end{enumerate}

    \end{answer}


  \item One of the most significant contributions from the developers
    of ALGOL 60 also limited the success of that language. What was
    that contribution?

  \begin{answer}

    "Ironically, one of the most important contributions to computer
    science associated with ALGOL 60, BNF, was also a factor in its
    lack of acceptance. Although BNF is now considered a simple and
    elegant means of syntax descrip- tion, in 1960 it seemed strange
    and complicated."  pg. 57 (10th edition)

    \end{answer}

  \item What problem were the creators of Common LISP trying to solve?

  \begin{answer}

    The developers of Common LISP were trying to solve issues with
    portability among programs written in various dialects.

    \end{answer}

  \item What is an ambiguous context free grammar?

  \begin{answer}

    "A grammar that generates a sentential form for which there are
    two or more distinct parse trees is said to be ambiguous"

    \end{answer}

  \item Contrast the complexity of algorithms that can parse strings
    that conform to the most general kinds of context free grammars
    and the complexity of the algorithms that can parse strings that
    conform to the grammars of programming languages?

  \begin{answer}

    Parsing algorithms for unambiguous grammars are complicated and
    inneficient. In fact, " the complexity of such algorithms is
    $O(n^3)$"(pg. 180). On the other hand, the algorithms used for
    context free grammars are closer to the level of $O(n)$, which means
    the time they take is linearly related to the length of the string
    to be parsed. This is vastly more efficient than $O(n^3)$
    algorithms.

    \end{answer}

  \item Java represents characters with Unicode. It is the first
    widely used programming language with this feature. What is the
    significance of this feature?

  \begin{answer}

    The previously used ASCII was becoming obsolete with the
    globalization of business and the need for computers to
    communicate around the world. Java quickly becomes a global coding
    language based on it's acceptance of unicode - which includes
    (among other things) the Cryllic alphabet.

    \end{answer}

  \item How does the binary coded decimal type differ from the
    floating point type?

  \begin{answer}

    "Decimal types have the advantage of being able to precisely store
    dec- imal values, at least those within a restricted range, which
    cannot be done with floating-point"

    \end{answer}

  \item Identify a user-defined ordinal type in the Java programming
    language.

  \begin{answer}

    "There are two user-defined ordinal types that have been supported
    by programming languages: enumeration and subrange."

    \end{answer}

  \item Mathematicians and programmers might have different ideas
    about the precedence of Boolean operators. Explain.

  \begin{answer}

    Because in programming, arithmetic expressions can be the operands
    of relational expressions, and relational expressions can be the
    operands of Boolean expressions. For example the '=' sign - in
    mathmatics is signifies that two sides of an equation are the same
    value, wheras in programming it signifies the changing of value of
    a variable (or other things).

    \end{answer}

  \item Programmers should use \verb+===+ rather than \verb+==+ to
    test the equality of the values of two expressions in JavaScript. Why?

  \begin{answer}

    Because the '==' operator uses coercions to achieve equality,
    '===' is testing for it.

    \end{answer}

  \item Describe a hazard of allowing short-circuited evaluation
    of expressions and side effects in expressions at the same time.

  \begin{answer}

    "Suppose that short-circuit evaluation is used on an expression
    and part of the expres- sion that contains a side effect is not
    evaluated; then the side effect will occur only in complete
    evaluations of the whole expression."

    \end{answer}

  \item Briefly describe the three steps in the mark-sweep algorithm
    for garbage collection.

  \begin{answer}

    First, all cells in the heap have their indicators set to indicate
    they are garbage.  Second, Every pointer in the program is traced
    into the heap, and all reachable cells are marked as not being
    garbage Third, All cells in the heap that have not been
    specifically marked as still being used are returned to the list
    of available space
    \end{answer}

  \item What led Yukihiro Matsumoto to create the Ruby programming language?

  \begin{answer}

    "The motivation for Ruby was dissatisfaction of its designer with
    Perl and Python. Although both Perl and Python support
    object-oriented programming,14 nei- ther is a pure object-oriented
    language, at least in the sense that each has primi- tive
    (nonobject) types and each supports functions."(pg. 100)

    \end{answer}

  \item What did Microsoft aim to achieve with its development of the
    \verb+C#+ language?

  \begin{answer}

   "The purpose of \verb+C#+ is to provide a language for component-based
    software development, specifically for such development in the
    .NET Framework. In this environment, components from a variety of
    languages can be easily com- bined to form systems"

    \end{answer}

  \end{enumerate}



\section{More questions for discussion and review.}

\begin{enumerate}
  \item The design of which machine influenced the design
    of the control statements in FORTRAN?
    
\begin{answer}

    IBM 704
    
\end{answer}

  \item How many different kinds of control statements
    must the designer of a programming language include
    in a language?

\begin{answer}

    It's possible to use one if you are using the (GOTO) statement, but otherwise the minimum is two control statements. 
    
\end{answer}

  \item What is the one question that applies in the
    design of all statements that allow selection or
    iteration?

\begin{answer}

    Whether or not the control structure should have multiple entries.
    
\end{answer}

  \item What is an advantage of requiring that
    the \textbf{then} and \textbf{else} clauses of
    an \textbf{if} statement be compound statements?
    
\begin{answer}

    Compound statements increase readability and writability when using nested loops and selector statements. Otherwise it would get pretty ugly pretty fast...
    
\end{answer}

  \item How does the \textbf{switch} statement in C\#
    differ from the \textbf{switch} statement in Java?
    
    \begin{answer}
    
    C\# allows execution of more than one segment, also the control expressions can be strings as well as case statements. In Java, the switch statement does not allow case expressions anywhere except the top level of the switch.
    
\end{answer}

  \item Distinguish between 2 statements in Ruby
    that correspond to Java's \textbf{switch} statement.
    
\begin{answer}

    Case expressions are the Ruby equivalent of Java's \textbf{switch} statement. One is semantically similar to nested if statements with case - when - then. The other is with boolean expressions being evaluated one at a time from top to bottom. The value of this case expression is equivalent to the value of the \textbf{then} statement corresponding to the first \textit{true} \textbf{when} statement. 
    
\end{answer}

  \item Features of a programming language sometimes persist
    longer than a feature of computing hardware that inspired
    and supported that part of the language's design.
    Similarly, features of hardware sometimes persist longer
    than some parts of a language's design that were created
    to take advantage of that feature in hardware.
  
\begin{answer}

    \begin{enumerate}
        \item The IBM 704 influenced the design of control statements which
        are still used today, and prompted the development of Fortran.
    
	   \item The best example of a feature that has persisted is the
           Von Neumann Architecture. In a von Neumann computer, both
           data and programs are stored in the same memory. The CPU,
           which executes instructions, is separate from the
           memory. Therefore, instructions and data must be
           transmitted, or piped, from memory to the CPU. Results of
           operations in the CPU must be moved back to memory. Nearly
           all digital computers built since the 1940s have been based
           on the von Neumann architecture.
    
        \item The Algol standard used several different syntaxes which,
        among other things, allowed Europeans to use a comma to denote
        a decimal point, while Americans could continue to use a
        period.

        \item The \textbf{register} keyword in C is a hint to the compiler
        that a variable will be used repeatedly, and so it should be
        stored in the CPU rather than in memory. However, modern
        compilers are far better at optimization than programmers, so
        this keyword is outdated and unnecessary.
        
     \end{enumerate}

\end{answer}

  \item Who most famously warned of the dangers of using the
    \textbf{goto} statement? What did Donald Knuth have to
    say about the use of the \textbf{goto} statement?
    
\begin{answer}

    Edsger Dijkstra noted “The goto statement as it stands is just too
    primitive; it is too much an invitation to make a mess of one’s
    program.” Donald Knuth argued there were occasions when the
    efficiency of the goto outweighed its harm to readability.
    
\end{answer}


  \item Describes Ada's \textbf{for} loop. Are there some
    kinds of iteration that might be easier in Ada than
    in Java? Easier in Java than in Ada?

\begin{answer}

    The for loop looks like this

\begin{lstlisting}{}
for variable in [reverse] discrete_range loop … end loop; 
\end{lstlisting}

Ada’s for loop can use any ordinal type variable for its
counter. Arrays with ordinal type subscripts can be conveniently
processed. (261) For loops in Java are more flexible - can have
infinite loops, change loop variable inside body, etc. but is
potentially more confusing to read.

\end{answer}


  \item What does it mean to say that the guarded commands
    of Ada are non-deterministic?
    
\begin{answer}

    Guarded commands in Ada are nondeterministically chosen for
    execution when more than one of the statements are evaluated to
    true. This means that if there are three guarded statements and
    two of the three evaluate to true, then each time the program will
    use one of the two statements. It will not always use the one that
    appears first nor the one that appears last, but rather it will
    choose between them non-deterministically or randomly.
    
\end{answer}


  \item The header files in a C program contain function
    prototypes. What is a function prototype?

\begin{answer}

    A function prototype is a function declaration that gives the
    function's name and type signature, but does not specify the
    function body. It is also referred to as a function interface at
    times. In other languages these are uncommon because subprograms
    do not need declarations since they do no need to be defined
    before they are called.
    
\end{answer}


  \item Every method in a Ruby program belongs to a class.
    A programmer can place a definition of a method inside
    the definition of a class or outside of the definition
    of any class that the programmer writes. To which class
    does the method belong in the second case?
 
 \begin{answer}
 
    If a method is defined outside of the definition of any class that
    the programmer writes then the method belongs to the root object,
    \textbf{Object}.
    
\end{answer}

  \item Distinguish between positional and keyword parameters.
  
\begin{answer}

  Positional parameters are bound based on the order in which they are
  given to the function. Keyword parameters are when the name of the
  formal parameter to which an actual parameter is to be bound is
  specified with the actual parameter in a call. This allows them to
  be input in any order.
  
\end{answer}


  \item Ruby blocks are closures. What does that mean?
  
\begin{answer}

  A closure is an anonymous function which can be passed as a
  parameter. Also, Ruby doesn't nest scope, so variables defined in a
  method are not accessible outside of the method (say, in the class
  which defines them).
  
\end{answer}


  \item What is a pure function?

\begin{answer}

A \textbf{pure function} is a \textbf{function} where the return value
is only determined by its input values, without observable side
effects.

\end{answer}

  \item Some languages give programmers means to define
    both functions and procedures. Java does not. Is that
    a serious limitation?

\begin{answer}

It does not seem to be a serious limitation, it just means that
performing procedures in Java is more verbose because it requires that
the programmer instantiate the class with any needed methods.

\end{answer}


  \item Declarations of formal parameters in an Ada procedure
    can include, in addition to the names and types of the
    parameters, reserved words that do not appear in declarations
    in Java programs. 
    What is the purpose of those reserved words?

\begin{answer}

    Ada allows the programmer to specify \textit{in} mode,
    \textit{out} mode, and \textit{inout} mode for each formal
    parameter.  This means that they can receive data from the actual
    parameter, they can send data to the actual parameter, or they can
    do both.
    
\end{answer}
 
  \item The C language imposes a constraint upon programmers
    who want to pass a multidimensional array to a function.
    What is the constraint? How did the design of the Java
    programming language eliminate that constraint for 
    programmers who use that language?

\begin{answer}

In C, programmers are required to pass the length of an array to a
method. Specifying the length of an array (like in C) is unnecessary
because Java arrays store a pointer to an address containing their
length.

\end{answer}

  \item An activation record contains a return
    address, a dynamic link, parameters, and
    local variables.
  \begin{enumerate}
    \item To what does the return address point?
    \item To what does the dynamic link point?
    \end{enumerate}
  
  \begin{answer}
  
    The return address usually consists of a pointer to the
    instruction following the call in the code segment of the calling
    program unit.  The dynamic link points to the base of the
    activation record instance of the caller.
    
\end{answer}


  \item The stack will contain multiple activation
    records for a single subprogram under what
    circumstances?
    
\begin{answer}

    In a recursive program there can be multiple activation records
    (although they will be incomplete) for a single subprogram.
    
\end{answer}


  \item How (or why?) does the LIFO protocol apply to
    calls to and returns from subprograms?
 
 \begin{answer}
 
    LIFO allows for subprograms to be nested within one another and
    for subprograms to be used as parameters within other subprograms.
    
\end{answer}

  \item Which important development in computer architecture
    has changed the way that the stack is used in some
    systems for facilitating calls to and returns from
    subprograms?

\begin{answer}

    RISC (reduced instruction set computing) machines have parameters
    passed in registers in their compilers because RISC machines have
    more registers than CISC (complex instruction set computing)
    machines. Chapter 10 assumes parameters are passed in the stack
    though, as they had been in CISC machines.
    
\end{answer}

  \item A dynamic chain contains a history of what?

\begin{answer}

  A dynamic chain represents the dynamic history of how the execution
  got to its current position, which is always in the subprogram code
  whose activation record instance is on top of the stack.
  
\end{answer}

  \item Which two numbers are needed to compute
    the address of a local variable in a subprogram?

\begin{answer}

To compute the address of a local variable you need the (chain\_offset,
local\_offset) pair.

\end{answer}

  \item How does a Ruby module differ from a class?
  
  \begin{answer}
  
  Modules are unlike classes in that they cannot be instantiated or
  subclassed and do not define variables. Methods that are defined in
  a module include the module’s name in their names.
  
\end{answer}

  \item Memory for variables can be allocated on the heap
    and on the stack. In which place or places is memory
    allocated for objects in C++? in Java?
    
\begin{answer}

    In C++, variables can be allocated to the heap either by making
    them “static” or by allocating memory with the keyword
    “new”. However, variables which are initialized during the
    execution of a function are allocated to the stack. Java behaves
    the same way.
    
\end{answer}

  \item What problems were solved by the addition
    of genericity to Java?
    
\begin{answer}

    Generics allow a type or method to operate on objects of various
    types while providing compile-time type safety.
    
\end{answer}

  \item What is the purpose of the static chain?
  
\begin{answer}

  The static chain is a path of pointers which go from each function
  to its parent. They allow child subprograms to use variables which
  are local to their parent, grandparent, or farther up, without
  needing to copy those variables to the call stack.
  
\end{answer}

  \item What is a singleton?
  
\begin{answer}

  A singleton is a class which provides a global access point to a
  single instance. This is useful for tasks which only need one point
  of access, like a file system.
  
\end{answer}

  \item What are the two parts of the definition 
    of an abstract data type?
    
\begin{answer}

A type definition which allows program units to declare variables of
the type but hides the representation of objects of the type. 2. A set
of operations for manipulating objects of the type.

\end{answer}

  \end{enumerate}


