
\chapter{Sam Chalkley}

\begin{enumerate}
  \item Most programming languages require the use of brackets to
    enclose the index in a reference to an element of an array.
  \begin{enumerate}
    \item Identify a language the requires the use of parentheses
      to enclose the index in a reference to an element of an array.
    \item Why did the designers of the language choose parentheses
      rather than brackets?
    \end{enumerate}

  \begin{answer}
    the IBM 704 system influenced the design of control statements in FORTRAN
   

  \begin{enumerate}
    \item the language that requires the use of parentheses to enclose
      an index is ada.
    \item the designers chose parentheses to enclose subscripts so
      there would be uniformity between array references and function
      calls in expressions.
    \end{enumerate}

    \end{answer}
    
  \item What is the relationship between a lexeme and a token?

  \begin{answer}

    the lexemes of a programming language include numeric literals
    operators and special words. these are partitioned into
    groups. the token is the name that represents each lexeme
    group. (a token of a language is a category of its lexemes.)

    \end{answer}

  \item
  \begin{enumerate}
    \item What kind of symbols are found at the internal nodes of a
      parse tree?
    \item What kind of symbols are found at the leaves of a parse tree?
    \end{enumerate}

  \begin{answer}

  \begin{enumerate}
    \item nonterminal symbols are at every internal node
    \item leaves of the parse tree is labeled with a terminal symbol
    \end{enumerate}

    \end{answer}


  \item One of the most significant contributions from the developers
    of ALGOL 60 also limited the success of that language. What was
    that contribution?

  \begin{answer}

   This is what is called static scopping it is a method of binding
   names to nonlocal variables. the scope of the variable can be
   statically determined prior to execution.
    \end{answer}

  \item What problem were the creators of Common LISP trying to solve?

  \begin{answer}

  it was created to solve lack of portability among programs written
  in the various dialects

    \end{answer}

  \item What is an ambiguous context free grammar?

  \begin{answer}
    a grammarthat generates a sentential form where there are two or
    more distinct parse trees.
        \end{answer}

  \item Contrast the complexity of algorithms that can parse strings
    that conform to the most general kinds of context free grammars
    and the complexity of the algorithms that can parse strings that
    conform to the grammars of programming languages?

  \begin{answer}

    the conplexity for parsing strings that conform to general kinds
    of context free grammers is $O(n^3)$ which is alot and takes
    awhile. while parsing strings that conform to grammars have a
    complexity of $O(n)$ which is much faster. this indicates that it is
    linear. and much more efficient

   ************************* Write your answer here.

    \end{answer}

  \item Java represents characters with Unicode. It is the first
    widely used programming language with this feature. What is the
    significance of this feature?

  \begin{answer}

  Unicode is standard for character incoding. this allowed computers
  to communicate with comuters around the world. this helped with
  globalization of business because of the demand of computers talking
  to comuters around the world.

    \end{answer}

  \item How does the binary coded decimal type differ from the
    floating point type?

  \begin{answer}

  uses binary codefor the decimal digits ( much like strings.) this
  allows precision storing of decimal values. supports business
  proccessing etc.  on the other hand floating points are only
  approximations for many real values. has some problems like loss of
  accuracy throug arithmetic operations

    \end{answer}

  \item Identify a user-defined ordinal type in the Java programming
    language.

  \begin{answer}

   this is an enumeration. 

    \end{answer}

  \item Mathematicians and programmers might have different ideas
    about the precedence of Boolean operators. Explain.

  \begin{answer}

   in math booleans or/ and must have equal precedence.  however with
   computer science, specifically in c we establish a higher
   precedence in and than or.

    \end{answer}

  \item Programmers should use \verb+===+ rather than \verb+==+ to
    test the equality of the values of two expressions in JavaScript. Why?

  \begin{answer}

    because this === prevents operands from being coerced.in this case
    === with a string and number would turn out false.

    \end{answer}

  \item Describe a hazard of allowing short-circuited evaluation
    of expressions and side effects in expressions at the same time.

  \begin{answer}

   short circut evaluation can be hazardus when used on and expression
   and part of an expression that contains a side effect is not
   evaluated.  for example $(a > b) || ((b++ / 3)$ if the programmer
   assumed b would change every time and a was not greater than b the
   program would fail.

    \end{answer}

  \item Briefly describe the three steps in the mark-sweep algorithm
    for garbage collection.

  \begin{answer}

  1) all cells in the heap have their indicators set to indicate they
  are garbage 2) marking phase, every pointer is traced into the heap
  and all reachable cells are marked as not being garbage 3)sweep
  phase, all cells in the heap that have not been specifically marked
  as still being used are returned to the list of available space.

    \end{answer}

  \item What led Yukihiro Matsumoto to create the Ruby programming language?

  \begin{answer}

   the motivation for ruby was dissatisfaction of its designer with
   perl and python.  niether was a pure object-oriented language in
   the sense that each hase primitive types and each supports
   functions.

    \end{answer}

  \item What did Microsoft aim to achieve with its development of the
    C\# language?

  \begin{answer}

   the C\# designers obviously disagreed with this wholesale removal of
   featurs that java did not include. it included all but the multiple
   inheritance. they wanted to have a language for component based
   software development (looking alot at .NET)


    \end{answer}

  \end{enumerate}



\section{More questions for discussion and review.}

\begin{enumerate}
  \item The design of which machine influenced the design
    of the control statements in FORTRAN?
    \begin{answer}
    IBM 704 system influenced the design of the control statments in FORTRAN
    \end{answer}
  \item How many different kinds of control statements
    must the designer of a programming language include
    in a language?
    \begin{answer}
      while it is possible to use only one goto, the minimum number of control statements in a language that doesnt use goto is two. first for choosing between control flow paths, and the other for logically controlled iterations.
      \end{answer}

  \item What is the one question that applies in the
    design of all statements that allow selection or
    iteration?
    \begin{answer}
      should the control structure have multiple entries?
      \end{answer}

  \item What is an advantage of requiring that
    the \textbf{then} and \textbf{else} clauses of
    an \textbf{if} statement be compound statements?

    \begin{anwer}
      It adds readability, as well as ensuring the encapsulation of just what you want in your if statment
    \end{anser}


  \item How does the \textbf{switch} statement in C\#
    differ from the \textbf{switch} statement in Java?
    
    \begin{answer}
      in java the switch statement does not allow case expressions anywhere exept the top level in the body of the switch. C/# 
allows the execution of more than one segment. In c/# the control exressions as well as case statements can be strings
    \end{answer}

  \item Distinguish between 2 statements in Ruby
    that correspond to Java's \textbf{switch} statement.

    \begin{answer}
     Case expressions are the ruby equivalent of java's \textbf{switch} statement. One is semantically similar to  nested if statements with case -when- then. the other is with boolean expressions being evaluated one at a time from top to bottom.
     the value of this case exression ends up as the first value with a true corresponding boolean.
      
    \end{answer}

  \item Features of a programming language sometimes persist
    longer than a feature of computing hardware that inspired
    and supported that part of the language's design.
    Similarly, features of hardware sometimes persist longer
    than some parts of a language's design that were created
    to take advantage of that feature in hardware.

    Give examples.

    \begin{answer}
      the IBM 704 influenced the design of control statements

  \item Who most famously warned of the dangers of using the
    \textbf{goto} statement? What did Donald Knuth have to
    say about the use of the \textbf{goto} statement?

    \begin{answer}
      Edsger Dijkstra- ``the goto statment as it stads is just too primitivel; it is too much an invitation to make a mess of one's program.''Donald Knuth argued there were occasions when the efficiency of the goto outweighed its harm to readability
      \end{answer}

  \item Describes Ada's \textbf{for} loop. Are there some
    kinds of iteration that might be easier in Ada than
    in Java? Easier in Java than in Ada?

    \begin{answer}
      with ada the counter variable cannot be changed in the body. the counter is edited in the body. you aslo gain flexability by
      allowing you to edit any part of the loop.
    \end{answer}

  \item What does it mean to say that the guarded commands
    of Ada are non-deterministic?
    \begin{answer}
      Guarded commands in Ada are nondeterministically chosen for execution when
      more than one of the statements are evaluated to be true, then each time the
      program will use one of the two statements. It will not always use the one 
      that appears last, but rather it will choose between them randomly.
    \end{answer}

  \item The header files in a C program contain function
    prototypes. What is a function prototype?

    \begin{answer}
      a function prototype is a function declaration that give the functions name
      and type signature, but does not specify the function body.
      It is also referred to as a function interface at times. in other languages 
      these are uncommon becuase subprograms do not need declarations since they do
      not need to be defined before they are called
    \end{answer}

  \item Every method in a Ruby program belongs to a class.
    A programmer can place a definition of a method inside
    the definition of a class or outside of the definition
    of any class that the programmer writes. To which class
    does the method belong in the second case?

    \begin{answer}
      if a method is defined outside of the definition of any class that the
      programmer writes then the method belongs to the root object, \textbt{object}
      
    \end{answer}

  \item Distinguish between positional and keyword parameters.

    \begin{answer}
      Positional parameters are bound based on the order in which they are given
      to the function. Keyword parameters are when the name of the formal parameter
      to which  an actual parameter is to be bound is specified with the actual 
      parameter in a call. this allows them to be input in any order.
    \end{answer}

  \item Ruby blocks are closures. What does that mean?

    \begin{answer}
      a closure is an anonymous function which can be passed as a parameter. also,
      Ruby doesnt nest scope, so variables defined in a method are not accessible 
      outside of the method.
    \end{answer}

  \item What is a pure function?
    
    \begin{answer}
     a pure function is a function where the return value is only determined by
     its input values, without observable side effects.
    \end{answer}

  \item Some languages give programmers means to define
    both functions and procedures. Java doe not. Is that
    a serious limitation?

    \begin{answer}
      it does not seem to be a serious limitation, it just means that the preforming
      procedures in Java are more verbose becase it requires that the programmer
      instantiate the class with any needed methods.
    \end{answer}

  \item Declarations of formal parameters in an Ada procedure
    can include, in addition to the names and types of the
    parameters, reserved words that do not appear in declarations
    in Java programs. 
    What is the purpose of those reserved words?

    \begin{answer}
      Ada allows the programmer to specify in mode, out mode, and inout mode for each formal parameter.  This means that they can receive data from the actual parameter, they can send data to the actual parameter, or they can do both.

    \end{anser}
 
  \item The C language imposes a constraint upon programmers
    who want to pass a multidimensional array to a function.
    What is the constraint? How did the design of the Java
    programming language eliminate that constraint for 
    programmers who use that language?

  \begin{answer}

    you need to tell the function how many columns in the array. java got rid of this allowing the array to tell the function 
    the number of columns
  \end{answer}

  \item An activation record contains a return
    address, a dynamic link, parameters, and
    local variables.
  \begin{enumerate}
    \item To what does the return address point?
    \item To what does the dynamic link point?
    \end{enumerate}

  \begin{answer}
 The return address usually consists of a pointer to the instruction following the call in the code segment of the calling program unit.  The dynamic link points to the base of the activation record instance of the caller.


  \end{answer}

  \item The stack will contain multiple activation
    records for a single subprogram under what
    circumstances?

    \begin{answer}
   
      In a recursive program there can be multiple activation records (although they will be incomplete) for a single subprogram.

    \end{answer}

  \item How (or why?) does the LIFO protocol apply to
    calls to and returns from subprograms?
    \begin{answer}
      LIFO allows for subprograms to be nested within one another and for subprograms to be used as parameters within other subprograms. (417)

    \end{answer}

  \item Which important development in computer architecture
    has changed the way that the stack is used in some
    systems for facilitating calls to and returns from
    subprograms?

    \begin{answer}
      RISC (reduced instruction set computing) machines have parameters passed in registers in their compilers because RISC machines have more registers than CISC (complex instruction set computing) machines. Chapter 10 assumes parameters are passed in the stack though, as they had been in CISC machines.

    \end{answer}

  \item A dynamic chain contains a history of what?
 \begin{answer}
   A dynamic chain represents the dynamic history of how the execution got to its current position, which is always in the subprogram code whose activation record instance is on top of the stack.

    \end{answer}


  \item Which two numbers are needed to compute
    the address of a local variable in a subprogram?
    \begin{answer}
      To compute the address of a local variable you need the (chain_offset, local_offset) pair.

    \end{answer}
    

  \item How does a Ruby module differ from a class?

    \begin{answer}
      Modules are unlike classes in that they cannot be instantiated or subclassed and do not define variables. Methods that are defined in a module include the module’s name in their names.

    \end{answer}

  \item Memory for variables can be allocated on the heap
    and on the stack. In which place or places is memory
    allocated for objects in C++? in Java?

    \begin{answer}
      In C++, variables can be allocated to the heap either by making them “static” or by allocating memory with the keyword “new”. However, variables which are initialized during the execution of a function are allocated to the stack. Java behaves the same way. 
    \end{answer}

  \item What problems were solved by the addition
    of genericity to Java?

    \begin{answer}
      Generics allow a type or method to operate on objects of various types while providing compile-time type safety.
    \end{answer}

  \item What is the purpose of the static chain?

    \begin{answer}
      The static chain is a path of pointers which go from each function to its parent. They allow child subprograms to use variables which are local to their parent, grandparent, or farther up, without needing to copy those variables to the call stack.

    \end{answer}

  \item What is a singleton?

    \begin{answer}
      A singleton is a class which provides a global access point to a single instance. This is useful for tasks which only need one point of access, like a file system.

    \end{answer}

  \item What are the two parts of the definition 
    of an abstract data type?

    \begin{answer}
      1. A type definition which allows program units to declare variables of the type but hides the representation of objects of the type. 2. A set of operations for manipulating objects of the type.

    \end{answer}

  \end{enumerate}


