
\chapter{Huong Nguyen}

\begin{enumerate}
  \item Most programming languages require the use of brackets to
    enclose the index in a reference to an element of an array.
  \begin{enumerate}
    \item Identify a language the requires the use of parentheses
      to enclose the index in a reference to an element of an array.
    \item Why did the designers of the language choose parentheses
      rather than brackets?
    \end{enumerate}

  \begin{answer}

  \begin{enumerate}
    \item The languages are pre-90 Fortrans and PL/I
    \item Because there were no other suitable characters available at
      the time. Card punches did not include bracket characters.
    \end{enumerate}

    \end{answer}
    
  \item What is the relationship between a lexeme and a token?

  \begin{answer}

    Lexemes are the lowest-level syntactic units. A token of a
    language is a category of its lexemes.

    \end{answer}

  \item
  \begin{enumerate}
    \item What kind of symbols are found at the internal nodes of a
      parse tree?
    \item What kind of symbols are found at the leaves of a parse tree?
    \end{enumerate}

  \begin{answer}

  \begin{enumerate}
    \item Nonterminal symbols are found at the internal nodes of a parse trees.
    \item Terminal symbols are found at the leaves of a parse trees.
    \end{enumerate}

    \end{answer}


  \item One of the most significant contributions from the developers
    of ALGOL 60 also limited the success of that language. What was
    that contribution?

  \begin{answer}

    It was the use of BNF as the formal means of describing syntax. In
    1960 it seemed strange and complicated.

    \end{answer}

  \item What problem were the creators of Common LISP trying to solve?

  \begin{answer}

    During the 1970s and early 1980s, a large number of different
    dialects of LISP were developed and used. This led to the familiar
    problem of lack of portability among programs written in the
    various dialects. Common LIPS was created to rectify the situation
    by combining the features of several dialects of LISP, including
    Scheme, into a single language.

    \end{answer}

  \item What is an ambiguous context free grammar?

  \begin{answer}

    An ambiguous grammar is a grammar that generates a sentential form
    for which there are two or more distinct parse trees. Context-free
    grammars were developed by Noam Chomsky as a class of natural
    languages, but became the primary method through which the syntax
    of programming languages are described.

    \end{answer}

  \item Contrast the complexity of algorithms that can parse strings
    that conform to the most general kinds of context free grammars
    and the complexity of the algorithms that can parse strings that
    conform to the grammars of programming languages?

  \begin{answer}

Algorithms that can parse general kinds of context-free grammars have
a cubic time complexity. But the grammars of programming languages are
far less general and only a subset of general context-free
grammars. Algorithms to parse them have O(n), or linear time
complexity, which is more efficient.

    \end{answer}

  \item Java represents characters with Unicode. It is the first
    widely used programming language with this feature. What is the
    significance of this feature?

  \begin{answer}

    Unicode includes the characters from most of the world's natural
    languages. Before Unicode was created, the 8-bit code ASCII was
    used to code character data and became inadequate for
    communication with computers around the world.

    \end{answer}

  \item How does the binary coded decimal type differ from the
    floating point type?

  \begin{answer}

    Floating-points store only approximations for many real values,
    while decimals store precise decimal values up to a specific
    range. Floating-point values are represented as fractions and
    exponents, typically taking 32 bits. Binary-coded decimals store
    one to two digits per byte, since a digit takes at least 4 bits,
    so they take up a lot of storage.

    \end{answer}

  \item Identify a user-defined ordinal type in the Java programming
    language.

  \begin{answer}

    The java user-defined ordinal type is Enum. All enumeration types
    in Java are implicitly subclasses of the predefined class Enum.

    \end{answer}

  \item Mathematicians and programmers might have different ideas
    about the precedence of Boolean operators. Explain.

  \begin{answer}

    In the mathematics of Boolean algebras, the OR and AND operators
    must have equal precedence. However, the C-base languages assign a
    higher precedence to AND than OR.

    \end{answer}

  \item Programmers should use \verb+===+ rather than \verb+==+ to
    test the equality of the values of two expressions in JavaScript. Why?

  \begin{answer}

    \verb+===+ prevents Javascript operands from being coerced, and in
    effect checks for type equivalence. For example, ``7'' \verb+==+ 7
    evaluates to true in Javascript, while ``7'' \verb+===+ 7
    evaluates to false.

    \end{answer}

  \item Describe a hazard of allowing short-circuited evaluation
    of expressions and side effects in expressions at the same time.

  \begin{answer}

    Allowing both at the same time may lead to errors where the side
    effect is part of the second boolean expression, and is not
    evaluated due to the expression having been short-circuited after
    the first.

    \end{answer}

  \item Briefly describe the three steps in the mark-sweep algorithm
    for garbage collection.

  \begin{answer}

    First, all cells in the heap are marked as garbage. Second, every
    pointer is traced into the heap, and mark all cells they reach as
    not-garbage. Finally, all cells still marked as garbage are
    returned to the list of available space.

    \end{answer}

  \item What led Yukihiro Matsumoto to create the Ruby programming language?

  \begin{answer}

    Matsumoto was dissatisfied with Perl and Python, which were not
    pure object-oriented languages.

    \end{answer}

  \item What did Microsoft aim to achieve with its development of the
    C\# language?

  \begin{answer}

    As the flagship language of .NET and Microsoft, C\# was
    meant to allow .NET to combine components written in all 
    other .NET languages, such as Visual Basic and C++.

    \end{answer}

  \end{enumerate}













\section{More questions for discussion and review.}

\begin{enumerate}
  \item The design of which machine influenced the design
    of the control statements in FORTRAN?

    \begin{answer}

    \begin{enumerate}
    \item The IBM 704 system.
    \end{enumerate}

    \end{answer}


  \item How many different kinds of control statements
    must the designer of a programming language include
    in a language?

    \begin{answer}

    \begin{enumerate}
    \item Only GOTO is necessary but if you don't use GOTO 
      2 are necessary, 1 for branching and 1 for iteration.
    \end{enumerate}

    \end{answer}

  \item What is the one question that applies in the
    design of all statements that allow selection or
    iteration?
    \begin{answer}

    \begin{enumerate}
    \item Should the control structure have multiple entries?
    \end{enumerate}

    \end{answer}

  \item What is an advantage of requiring that
    the \textbf{then} and \textbf{else} clauses of
    an \textbf{if} statement be compound statements?

    \begin{answer}

    \begin{enumerate}
    \item The advantage is for writability and readability of nested statements.
    \end{enumerate}

    \end{answer}
  
  \item How does the \textbf{switch} statement in C\#
    differ from the \textbf{switch} statement in Java?

      \begin{answer}

    \begin{enumerate}
    \item C\# enforces executing only the first statement.
    \end{enumerate}

    \end{answer}

  \item Distinguish between 2 statements in Ruby
    that correspond to Java's \textbf{switch} statement.

      \begin{answer}

    \begin{enumerate}
    \item One is semantically similar to nested if statements with
      case - when - then.  The other is with boolean expressions being
      evaluated one at a time from top to bottom.
    \end{enumerate}

    \end{answer}

  \item Features of a programming language sometimes persist
    longer than a feature of computing hardware that inspired
    and supported that part of the language's design.
    Similarly, features of hardware sometimes persist longer
    than some parts of a language's design that were created
    to take advantage of that feature in hardware.

    Give examples.

    \begin{answer}
    \begin{enumerate}
    \item The IBM 704 influenced the design of control statements
      which are still used today.
    \end{enumerate}

    \end{answer}
  
  \item Who most famously warned of the dangers of using the
    \textbf{goto} statement? What did Donald Knuth have to
    say about the use of the \textbf{goto} statement?
    
   \begin{answer}
   \begin{enumerate}
    \item  Dijkstra
    \end{enumerate}

    \end{answer}

  \item Describes Ada's \textbf{for} loop. Are there some
    kinds of iteration that might be easier in Ada than
    in Java? Easier in Java than in Ada?

    \begin{answer}
    \begin{enumerate}
    \item  Ada's loop is: 
      \begin{lstlisting}{} 
      for variable in [reverse] discrete_range loop … end loop; 
      \end{lstlisting}

     \item  Arrays with ordinal type subscripts are easier to process in Ada.
       Loops with changes inside are easier in Java.
    \end{enumerate}

    \end{answer}
  
  \item What does it mean to say that the guarded commands
    of Ada are non-deterministic?

    \begin{answer}
    \begin{enumerate}
    \item Guarded commands in Ada are nondeterministically chosen for
      execution when more than one of the statements are evaluated to
      true. This means that if there are three guarded statements and
      two of the three evaluate to true, then each time the program
      will use one of the two statements. It will not always use the
      one that appears first nor the one that appears last, but rather
      it will choose between them non-deterministically or randomly.
    \end{enumerate}

    \end{answer}

  \item The header files in a C program contain function
    prototypes. What is a function prototype?
    \begin{answer}

    \begin{enumerate}
    \item It's a function declaration that gives the function's name
      and type signature, but does not specify the function body.
    \end{enumerate}

    \end{answer}
  
  \item Every method in a Ruby program belongs to a class.
    A programmer can place a definition of a method inside
    the definition of a class or outside of the definition
    of any class that the programmer writes. To which class
    does the method belong in the second case?

  
    \begin{answer}

    \begin{enumerate}
    \item If a method is defined outside of the definition of any
      class that the programmer writes then the method belongs to the
      root object
    \end{enumerate}

    \end{answer}
  \item Distinguish between positional and keyword parameters.

    \begin{answer}

    \begin{enumerate}
    \item Positional parameters are bound based on the order in which
      they are given to the function. Keyword parameters are when the
      name of the formal parameter to which an actual parameter is to
      be bound is specified with the actual parameter in a call. This
      allows them to be input in any order.
    \end{enumerate}

    \end{answer}
  \item Ruby blocks are closures. What does that mean?

  
    \begin{answer}

    \begin{enumerate}
    \item A closure is an anonymous function which can be passed as a
      parameter.
    \end{enumerate}

    \end{answer}

  \item What is a pure function?

  
    \begin{answer}

    \begin{enumerate}
    \item A function where the return value is only determined by its
      input values, without observable side effects.
    \end{enumerate}

    \end{answer}

  \item Some languages give programmers means to define
    both functions and procedures. Java doe not. Is that
    a serious limitation?

  
    \begin{answer}

    \begin{enumerate}
    \item No, it just means that performing procedures in Java is more
      verbose because it requires that the programmer instantiate the
      class with any needed methods.
 
    \end{enumerate}

    \end{answer}

  \item Declarations of formal parameters in an Ada procedure
    can include, in addition to the names and types of the
    parameters, reserved words that do not appear in declarations
    in Java programs. 
    What is the purpose of those reserved words?
 
  
    \begin{answer}

    \begin{enumerate}
    \item Ada allows the programmer to specify \textit{in} mode,
      \textit{out} mode, and \textit{inout} mode for each formal
      parameter.  This means that they can receive data from the
      actual parameter, they can send data to the actual parameter, or
      they can do both.
    \end{enumerate}

    \end{answer}

  \item The C language imposes a constraint upon programmers
    who want to pass a multidimensional array to a function.
    What is the constraint? How did the design of the Java
    programming language eliminate that constraint for 
    programmers who use that language?

  
    \begin{answer}

    \begin{enumerate}
    \item In C, programmers are required to pass the length of an
      array to a method.  Specifying the length of an array (like in
      C) is unnecessary because Java arrays store a pointer to an
      address containing their length.
 
    \end{enumerate}

    \end{answer}

  \item An activation record contains a return
    address, a dynamic link, parameters, and
    local variables.
  \begin{enumerate}
    \item To what does the return address point?
    \item To what does the dynamic link point?
    \end{enumerate}

  
    \begin{answer}

    \begin{enumerate}
    \item The return address usually consists of a pointer to the
      instruction following the call in the code segment of the
      calling program unit.
    \item The dynamic link points to the base of the activation record
      instance of the caller.
    \end{enumerate}

    \end{answer}

  \item The stack will contain multiple activation
    records for a single subprogram under what
    circumstances?

    \begin{answer}

    \begin{enumerate}
    \item If the program is recursive. 
    \end{enumerate}

    \end{answer}
  
  \item How (or why?) does the LIFO protocol apply to
    calls to and returns from subprograms?

  
    \begin{answer}

    \begin{enumerate}
    \item LIFO allows for subprograms to be nested within one another
 and for subprograms to be used as parameters within other subprograms.  
    \end{enumerate}

    \end{answer}

  \item Which important development in computer architecture
    has changed the way that the stack is used in some
    systems for facilitating calls to and returns from
    subprograms?

    \begin{answer}

    \begin{enumerate}
    \item RISC (reduced instruction set computing) machines have
      parameters passed in registers in their compilers because RISC
      machines have more registers than CISC (complex instruction set
      computing) machines.
 
    \end{enumerate}

    \end{answer}
  
  \item A dynamic chain contains a history of what?

  
    \begin{answer}

    \begin{enumerate}
    \item It's the the dynamic history of how the execution got to its
      current position.
 
    \end{enumerate}

    \end{answer}

  \item Which two numbers are needed to compute
    the address of a local variable in a subprogram?

  
    \begin{answer}

    \begin{enumerate}
    \item chain offset and local offset.
    \end{enumerate}

    \end{answer}
  \item How does a Ruby module differ from a class?

  
    \begin{answer}

    \begin{enumerate}
    \item Ruby models cannot be instantiated or subclassed and do not
      define variables.  Also, methods that are defined in a module
      include the module’s name in their names.
    \end{enumerate}

    \end{answer}

  \item Memory for variables can be allocated on the heap
    and on the stack. In which place or places is memory
    allocated for objects in C++? in Java?

  
    \begin{answer}

    \begin{enumerate}
    \item In both C++ and Java, variables can be allocated to the heap
      either by making them “static” or by allocating memory with the
      keyword “new”.  However, variables which are initialized during
      the execution of a function are allocated to the stack.
    \end{enumerate}

    \end{answer}

  \item What problems were solved by the addition
    of genericity to Java?

  
    \begin{answer}

    \begin{enumerate}
    \item A type or method can operate on objects of various types
      with compile-time type safety.
    \end{enumerate}

    \end{answer}
  \item What is the purpose of the static chain?


    \begin{answer}

    \begin{enumerate}
    \item The static chain is a path of pointers which go from each
      function to its parent.  They allow child subprograms to use
      variables which are local to their parent, grandparent, or
      farther up, without needing to copy those variables to the call
      stack.
 
    \end{enumerate}

    \end{answer}
  \item What is a singleton?

  
    \begin{answer}

    \begin{enumerate}
    \item A class which provides a global access point to a single instance.  
    \end{enumerate}

    \end{answer}

  \item What are the two parts of the definition 
    of an abstract data type?

    \begin{answer}

    \begin{enumerate}
    \item A type definition which allows program units to declare
      variables of the type but hides the representation of objects of
      the type, and a set of operations for manipulating objects of
      the type.
 
    \end{enumerate}

    \end{answer}

  \end{enumerate}



\section{More questions for discussion and review.}

\begin{enumerate}
  \item The design of which machine influenced the design
    of the control statements in FORTRAN?

  \item How many different kinds of control statements
    must the designer of a programming language include
    in a language?

  \item What is the one question that applies in the
    design of all statements that allow selection or
    iteration?

  \item What is an advantage of requiring that
    the \textbf{then} and \textbf{else} clauses of
    an \textbf{if} statement be compound statements?

  \item How does the \textbf{switch} statement in C\#
    differ from the \textbf{switch} statement in Java?

  \item Distinguish between 2 statements in Ruby
    that correspond to Java's \textbf{switch} statement.

  \item Features of a programming language sometimes persist
    longer than a feature of computing hardware that inspired
    and supported that part of the language's design.
    Similarly, features of hardware sometimes persist longer
    than some parts of a language's design that were created
    to take advantage of that feature in hardware.

    Give examples.

  \item Who most famously warned of the dangers of using the
    \textbf{goto} statement? What did Donald Knuth have to
    say about the use of the \textbf{goto} statement?

  \item Describes Ada's \textbf{for} loop. Are there some
    kinds of iteration that might be easier in Ada than
    in Java? Easier in Java than in Ada?

  \item What does it mean to say that the guarded commands
    of Ada are non-deterministic?

  \item The header files in a C program contain function
    prototypes. What is a function prototype?

  \item Every method in a Ruby program belongs to a class.
    A programmer can place a definition of a method inside
    the definition of a class or outside of the definition
    of any class that the programmer writes. To which class
    does the method belong in the second case?

  \item Distinguish between positional and keyword parameters.

  \item Ruby blocks are closures. What does that mean?

  \item What is a pure function?

  \item Some languages give programmers means to define
    both functions and procedures. Java doe not. Is that
    a serious limitation?

  \item Declarations of formal parameters in an Ada procedure
    can include, in addition to the names and types of the
    parameters, reserved words that do not appear in declarations
    in Java programs. 
    What is the purpose of those reserved words?
 
  \item The C language imposes a constraint upon programmers
    who want to pass a multidimensional array to a function.
    What is the constraint? How did the design of the Java
    programming language eliminate that constraint for 
    programmers who use that language?

  \item An activation record contains a return
    address, a dynamic link, parameters, and
    local variables.
  \begin{enumerate}
    \item To what does the return address point?
    \item To what does the dynamic link point?
    \end{enumerate}

  \item The stack will contain multiple activation
    records for a single subprogram under what
    circumstances?

  \item How (or why?) does the LIFO protocol apply to
    calls to and returns from subprograms?

  \item Which important development in computer architecture
    has changed the way that the stack is used in some
    systems for facilitating calls to and returns from
    subprograms?

  \item A dynamic chain contains a history of what?

  \item Which two numbers are needed to compute
    the address of a local variable in a subprogram?

  \item How does a Ruby module differ from a class?

  \item Memory for variables can be allocated on the heap
    and on the stack. In which place or places is memory
    allocated for objects in C++? in Java?

  \item What problems were solved by the addition
    of genericity to Java?

  \item What is the purpose of the static chain?

  \item What is a singleton?

  \item What are the two parts of the definition 
    of an abstract data type?

  \end{enumerate}


